\documentclass{article}
\usepackage{graphicx} % Required for inserting images

\title{On the Investigation of Time}
\author{Colin McHugh}
\date{XX July 2025}

\begin{document}

\maketitle

\section{Introduction}
The classical conception of time, proposed in various theories, holds that time is primitive (independent or existing in the background); for if it were not primitive, it seems there cannot exist flux (change). This view, however, faces inadequacy, as it cannot justify simple propositions regarding time. Specifically, as will be used heavily later, the premiss that time is primitive cannot provide formal justification for asserting any difference between two otherwise identical states. In what shall follow, therefore, I shall demonstrate, by means of this inability to distinguish identical states, that time is indeed not primitive, but derivative from change. Once this is proved in §2, it will, by extension of the non-primitivity of time, be proved in section §3 that time is system-dependent, meaning different systems will have different timelines. 

\section{The Non-primitivity of Time}
In what follows, the proposition that time exists by virtue of motion alone shall be deduced. First, however, let us – in order to inspire our intuition – suppose a proposition (1) which posits ‘today is Tuesday.’ Consider immediately a second proposition (2) which posits an identical statement. The consideration, then, becomes whether these statements refer to the same Tuesday. By referencing ‘Tuesday,’ one can be referencing a vast array of days, as (1) and (2) are left sufficiently ambiguous. It can therefore be inquired: What formal rule (or ‘program’) could assign different indices to (1) and (2)?  Surely, there must be more to this act of distinction than a simple definition of their times. For if it were imagined that a world which, by the following Tuesday, were exactly the same as that of this Tuesday, there exists no program – without knowing a time index – whereby one can say the propositions are or are not identical. The important result shown above is that, in the absence of a timeline, certain moments become sufficiently ambiguous, leaving only the physical content (or properties) of such moments to be considered for analyzing identity between propositions such as ‘now is Wednesday’ or ‘tomorrow is Monday.’
More narrowly, let us consider the propositions ‘today is today’ and ‘today is tomorrow’ to be identical in all except the time at which they occur. If, truly, tomorrow were simply a precise repeat of every instance of today, then could today truly be considered different from tomorrow? The only apparent difference in such a case is the definitional one. That is, tomorrow cannot be identical to today because, on a broader scope of time, they do not occur in the same instant. 

However, the above explanation cannot be satisfactory, as it references the same time placed in question. That is, in order to suppose there is a distinction between an otherwise identical today and tomorrow, we must use a timeline, the very object of this investigation. Suppose the successive scope of time used to differentiate our definitions of ‘tomorrow’ and ‘today’ is now the object of our consideration, a set period of time that supposedly frames the events – in this instance, tomorrow and today – that occur in its bounds. In other words, we are to imagine a duration of time that begins with today and ends with tomorrow. Our understanding of a timeline would thus be that by which the difference between today and tomorrow is maintained. Yet, our problem of time has simply been displaced, since now we may consider how we ought to distinguish between the difference of two identical ‘today-tomorrow’ timelines, which would require, in order to maintain the theory of an abstract, primitive time, yet another subsuming timeline, one which places the timelines in flux of each other. The problem thus repeats itself with every level of consideration.

Suppose – now in generalization of the presented instance – that moments A and B are perhaps capable of being differentiated by time. Time frame A and time frame B, for any scope of consideration, are now treated as non-identical, as – since they would otherwise be identical – they are defined to be at separate points in time. Following the supposition that A and B are otherwise identical, which means only time can differentiate them, we obtain the following. If A and B are distinct, then there exists an external timeline to differentiate them, and if there is an external timeline to differentiate them, then A and B are distinct. Hence, we have the lemma: There exists an external timeline to A and B if and only if A and B are distinct. It follows immediately that B is a change from A, since the moments must be distinct in at least some way; i.e., B is a state of change from A, even if just in its time index, supposing, of course, that time really is a faculty which precedes events. If this is true, then it also follows that a higher-order timeline, given by {{A, B}, {C, D}}, is also in flux, as is its higher-order time to the nth degree, which is absurd. This follows by mere variable substitution, for we can let the timeline {A, B} be one of any order, contingent only upon how we define A and B. This mere principle of substitution allows for the immediate result that a timeline of any order will rely on a higher-order one for time to exist as primitive. If this is true of any timeline, then it is true of all timelines. In other words, our lemma applies to timelines of all orders. Our final lemma is thus: For all orders of A and B, there exists an external timeline to A and B if and only if A and B are distinct. Therefore, since all possible orders of A and B require the ability to distinguish A and B via a higher-order timeline that contains them, there does not exist a timeline which can exist as an absolute background for all possible events. Thus it cannot be maintained that, when our system of consideration is restricted to some number of identical states, any time can exist whatever. ‘Time,’ in other words, can only mean something which is purely definitional, meaning something which is defined differently depending on the scope of one’s consideration. It is necessary, then, that motion – both in the logical and physical sense – determines time, not the converse. Consequently, one timeline cannot be privileged, and time is a byproduct of change; i.e., B comes temporally after A if and only if the properties of B need be logically necessitated by those of A. Hence, we have obtained the desired lemma stated in the beginning: time is indeed not primitive, but derivative from change.

\section{The Locality of Time}
Naturally, hereafter, it can be proven true, contingent upon the previous finding, that time is localised. Intuitively, first and foremost, let us suppose a man – we shall call him ‘Albert’ – is frozen in what he shall call ‘time,’ relative to the surrounding world – or the system defined by that which is not in Albert’s system. To Albert, who experiences no change, time does not pass. However, to the rest of the environment surrounding him, which is supposed to change, he seems to experience time, albeit with static change. Albert’s system, hence, has no time, while the system of another observer – we shall call him ‘Kurt’ – who is not Albert does and goes further to impose his timeline upon Albert, an imposition which yields the illusion of time in a static system. Hence Kurt will perhaps note that Albert’s net change over time is zero, and, more precisely, the rate of change of his net change never surpasses nullity. Thus, for Albert, ‘time’ as he should refer to it is frozen with respect to his isolated system, while ‘time’ as Kurt should refer to it is not so. Such illustrates the locality of time. 

While before is merely an intuitive demonstration, it is an attainable conviction that, by extension of the nature of time that has been previously obtained via contradiction of supposing otherwise, a localised time is necessary. We begin with what has already been demonstrated: time exists by virtue of change. Thus we can consider a system, which is defined by a bounded scope for physical consideration, that does not change whatsoever relative to a larger, subsuming system. Relative to itself, the system experiences no time, for there is no change. This much can be proved by simple extension of the above proposition. It shall, however, also follow that the larger, subsuming system experiences change, which seems, paradoxically, to include the first system. This may be deduced extensionally, since the system of supposition experiences change, implying that it experiences time. Of course, then, unless we are to suppose that the system of no change is both experiencing time and not experiencing time in the same fashion, we ought to suppose that the two timelines of the two systems respectively are distinct, thus reducing our proposition to ‘between two systems of different magnitudes of change, there shall be distinct timelines of each respective system.’ It is, as well, necessary that each system observing the other shall impose its timeline upon the other, which follows by restricting the scope to either or and attempting to represent the events of the one in the other. This, as shown above to be a special case in the intuitive sketch, is the locality of time. 

\section{Conclusion}
As a result of these findings, time can be described as two things: (1) An illusion given by change or motion, particularly in that the process of logical differentiation (change as described in the introductory footnote) is its cause. This, should it be aptly justified, provides a profound insight into the nature of physical phenomena. In particular, it seems that such phenomena exist as a history of logical propositions; i.e. the universe operates in a deductive system given by its first premises, and its laws merely follow them as a consequence. (2) A localised perception dependent upon how one arbitrarily defines a given system. Such a property should suffice to explain time as a purely definitional, arbitrary phenomenon, and by that virtue it is experienced differently depending on the collective system to which its definition is given meaning. It is clear, finally, that, as for the A-theory and B-theory of time, defining time as either equally real or once real is to assume ontology of time as at least a background entity driving change throughout all systems, which is irreconcilable with the suggestions of this work. However, both theories may, given slight adjustment in wording, still be plausible if they are to simply adjust to the view that time does not spur change in any respect, instead focusing on how the states themselves constitute time. Until now these theories were ones of time; hereafter, then, let them be theories of change.

\section{Objections and Replies}
It is possible, of course, that the primary demonstration is challenged in favour of a primitive time, as it may be asserted that the demonstration needlessly assumes that a timeline must be in flux by some higher-order timeline that contains it. Furthermore, it could be proposed that the usage of the phrase ‘in flux’ smuggles in the very conclusion the demonstration wishes to draw. A B-theorist, for example, could claim such things, in favour of the premise that time exists as simply another index of events. This objection, however, is inadequate in one critical account. In particular, the proposition that a timeline, when placed in yet another timeline, must be in flux is, when dismissed as arbitrary, contradictory. For the very investigation of this essay is that by which the cause of change is investigated. Recall that the necessity of assigning a timeline to the timeline began by assuming that time is a background element. Without supposing time is the way by which one distinguishes two otherwise identical events, there is no ability to properly claim time, the very thing that is supposed to create such a difference, is capable of distinguishing anything at all. Therefore, the objector must conclude the validity of the setup of the demonstration. Hence, without simply deferring the problem of distinguishing identical events to a higher timeline, the objector must either show the invalidity of the deferral process or conclude that all possible timelines rely on a higher one, meaning no such absolute timeline exists. In other words, rejecting the hierarchy of timelines is to the immediate detriment of the objector.
	
It may, as well, be further questioned whether the position advanced here is simply too unintuitive. Surely, it might be argued, time must be a primitive concept that supplies the very perception of change. If not, then we must explain how change occurs at all. In response, we can illustrate why the premise of a primitive time feels so intuitive. As rational agents, we reflect on time as it affects us. When we observe other systems, we impose our own timeline upon them, as demonstrated by the locality of time. But when we reflect on ourselves – a process unique to rational minds – we effectively treat our own system as if it were external, and thus implicitly ascribe to ourselves a primitive notion of time. In other words, self-reflection is the primary engine of the intuition that time is primitive. This is ultimately explained by the locality of time as discussed above.

\section{References}
Kant, Immanuel, Critique of Pure Reason, trans. Paul Guyer and Allen W. Wood (Cambridge: Cambridge University Press, 1998)

McTaggart, J.M.E., ‘The Unreality of Time’, Mind, 17 (1908), 457–474

\end{document}

